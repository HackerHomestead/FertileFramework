\chapter{Hardware}

Ah we start to get into the more literal hands on part of this book. The hardware! There are many um very many options to choose from to get you started on getting your “Hands On” learning experience with Computer hardware. I am going to make some recommendations to limit the choices a bit. This will be the hardware we will use in the book to for examples. I am going to attempt to use some examples that fall within the following attributes. 
a) Representative
b) Obtainable
c) Affordable 
d) Open – as in they will run Linux without much hassle
e) Somewhat portable
Generally we will be using hardware that is built with the x86 CPU architecture. We plan on a supplement covering the raspberry pi (ARM) and the TeaCup Tinker board (MIPS). But for now the very most common architecture is the Intel x86.
Hardware released in the last 10 or even 15 years is sufficient to as a learning platform. Plus it has the additional benefit to the environment if you re-use hardware that has been deemed “obsolete” by commercial interests, who at the end of the day are motivated to sell you the newest shiniest thing when the old thing still does the job quite well. 
As a Hacker in training you should also find some satisfaction for using some of these devices for presupposes they where not originally intended for. Breathing new life into a thing and using it to its fullest leaves you with a feeling satisfaction, not to mention more money in your wallet. Don’t be suckered into the hype those very same manufactures use hardware for 10+ years in production for no other reason than it can be quite costly to swap out a fleet of servers, laptops, thin clients, monitors, and all means of smaller devices that run enterprises.

\section{Recommended Computer Hardware}

Alright lets get to it, ...
