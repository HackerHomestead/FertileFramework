\chapter{Hardware}

Ah we start to get into the more literal hands on part of this book. The hardware! There are many um very many options to choose from to get you started on getting your “Hands On” learning experience with Computer hardware. I am going to make some recommendations to limit the choices a bit. This will be the hardware we will use in the book to for examples. I am going to attempt to use some examples that fall within the following attributes. 
a) Representative
b) Obtainable
c) Affordable 
d) Open – as in they will run Linux without much hassle
e) Somewhat portable
Generally we will be using hardware that is built with the x86 CPU architecture. We plan on a supplement covering the raspberry pi (ARM) and the TeaCup Tinker board (MIPS). But for now the very most common architecture is the Intel x86.
Hardware released in the last 10 or even 15 years is sufficient to as a learning platform. Plus it has the additional benefit to the environment if you re-use hardware that has been deemed “obsolete” by commercial interests, who at the end of the day are motivated to sell you the newest shiniest thing when the old thing still does the job quite well. 
As a Hacker in training you should also find some satisfaction for using some of these devices for presupposes they where not originally intended for. Breathing new life into a thing and using it to its fullest leaves you with a feeling satisfaction, not to mention more money in your wallet. Don’t be suckered into the hype those very same manufactures use hardware for 10+ years in production for no other reason than it can be quite costly to swap out a fleet of servers, laptops, thin clients, monitors, and all means of smaller devices that run enterprises.

\section{Recommended Computer Hardware}

Alright lets get to it, The examples in this book (2024) are recommended to be used on the following hardware or something like it. These devices while not only being inexpensive and obtainable also have a solid “built to last” design, are well supported by the open source community … and well since they have been out for a while you will be able to find plenty of documentation, forum posts, and other related information on how to get them to dace for you. 
\section{Laptops}
I chose both of these because they meet the minium attributes already mentioned but also they are obtainable on the used market (and our ebay store), plenty of replacement parts are available, they use modern SATA for storage and include wifi modules that are well supported by Linux, and both support Intel 32bit and 64bit modes and have built in DVD writers (Don’t underestimate the utility of optical media), and of this writing (2024) both are still supported to run and install the latest Linux distributions like Ubuntu 22.04.3 LTS (Supported until 2027).  Lastly I own them ;)

IBM Thinkpad T60 (2006)
Used Price $50 - $150
https://www.thinkwiki.org/wiki/Category:T60
4GB RAM * You will need to disable wayland, we will cover setup later. 

IBM Thinkpad X60 [Tablet] (2006)
Convertible tablet or Ultra-portable Laptop Formfactor
Meets the Free Software Foundation’s Requirements

Apple MacBook Pro 13 Inch (Mid 20210)
Used Price $50 - $150
https://everymac.com/systems/apple/macbook_pro/specs/macbook-pro-core-i5-2.4-aluminum-15-mid-2010-unibody-specs.html
4 or 8 GB of RAM, besides a CD/DVD burner it also comes with a full size SD card slot and some other good I/O
These intel Apple MacBooks Pros run Linux really well … what did you think I was going to suggest running OS X ;)
I also have an slightly older MacBook Air. Not a super fan of apple OS X but they sure can make a nice peace of hardware that stands up to the test of time.
Thin Clients
We are including some Thin Client options as these will be the most obtainable and affordable options. You can find them in lots of 5 or 10 +
Whats a “Thin Client” you ask? It is in essence the evolution of the “dumb terminal” a device designed specifically to connect to some server and only have the most basic of features locally while most of the heavy lifting was intended to be done in the back room, or remote data center. Think of cash registers, Airport and hotel check in desks, and even office environments where efficiencies are gained by centrally locating compute tasks on a server instead of providing workers with a full blown desktop … um like call centers, banking, stock trading investment firms, the list goes on and on … there are other benefits to using thin clients besides consolidation of compute and storage resources too such as security. If there is not much it can do or store locally we only have to secure access to the jewels in the back office. Not to mention they are generally cheaper and use older technology giving manufactures a last ditch effort to squeeze some profits out of a proven (and paid for) technology production line. 
Most thin clients produced since the year 2000 would be considered “graphical terminals” and really are just low-spec computers that include everything you need in a small footprint such as USB Ports, Flash Storage, Networking, Sound Cards, Graphics (sometimes even dual monitors). 
Something very interesting about ThinClients at least to the hacker is the processor included in these varies beyond just Intel or AMD … companies like VIA and Transmeta, also make even x86 compatible processors. And in these lower power devices these processors not only consume less power, but have some interesting extensions to support the embedded and mobile features as well. And this gives you an introduction to other spaces such as IoT and embedded domains. 
A fantastic resource for “last gen” Thin Client specs is 
https://www.parkytowers.me.uk/thin

Wyse Cx0 (~2010)
Something interesting about 

https://www.parkytowers.me.uk/thin/wyse/cx0/

Desktop
AMD FX(tm)-6300 Six-Core Processor and 16 or 32 GB of memory 1TB hard drive
I own 4 of these using an asus motherboard, I built a small vm lab and these where the compute nodes, there was an additional node for storage. They are really good processors the board is limited to 32GB of memory. The systems are 13 years old now. I still use one as a ZFS and general server. And my 15yr son uses his to play modern game title with the addition of a graphics card. 
Something else?
Notice a pattern? I generally don’t upgrade to a new computer generation but about every 10 years. I did just that yearly 2023 an ASUS ROG Strix from the prior year, this is my first computer much lesss a laptop with any kind of dedicated graphics processor. 
It features AMD Ryzen 9 5900HX with Radeon Graphics
Do you have to use these devices? Absolutely not, we are simply trying to help you choose from the barrage of options made in the last 20 years that will not only do the job of teaching you the basics of hacking the full stack, but also provide a lot of other productivity and yes games too. 
Personally my daily driver is a ASUS ROG Strix Gaming laptop with 64GB of memory and 2x2TB NvME drives and an NVIDA XXX. I did’t get it to play games on … at least not the games most people play ;)


