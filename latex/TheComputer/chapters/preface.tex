\chapter*{Preface}
\section*{About This Book}

This book is for those who want a practival guide for the digital universe. We provide a vertical slice approach to learning modern computers. The Computer will teach important concepts using examples that apply to large and small systems, from Mainframes and the "Cloud" all the way down to embedded devices that we interact with everyday.

I wrote this book for a few reasons

1) To offer an entry point for those who want to  look deeper into computers but don't know where to start. With all the advancements and just volumes of work done to advance the field it can be overwhelming to the new learner seeking to understand digital machines from the ground up.

2) When I was a boy your computer was expected ot come with a means to program it out of the box. This "creator culture" was baked right in. Imagine if your iPad foced you to learn a bit about it prior to smashing your finders for an instant dopamine response? Stand back for a second and ask yourself is the computer using you, or you it?

Of course not everyone needs to be a mechanic or an engineer in order to drive a car. I would suggest when the ratio of machanics dips under about 1 in 5 we might be at risk of being controlled by someone elses idea of what and how a car should be. Espessually when its such a critical part of our everyday lives.

3) As a young boy in the 1st grade in 1986 at the dawn of the "desktop revolution". I was dignosed with Dyslexia, Dysgraphia, and a shortage of working memory. These "learning disablitlites" that are now covered under the board umbrella of "nuro-divergint". School was difficult as this combo of neruo-processing differences where not all that compatable with standard teaching methods at the time. I was lucky enough that between my parents and teachers exploring new methods to help kids like me achive their potential I had technology thrown at me to help with writing, memory, reading, spelling, math, and early voice dictation. Computers helped me build a model of the world. A simple model that used levels of abstraction and offloaded my own need for route memorization and calculation tasks. As I grew this cross applied to understanding and modelting other difficult concepts like social interactions, philiphipy, econmics, etc. technology became not only a means of accomidation but a model, a microcusim for understanding the universe as it related to me.

So with 25+ years of professional experience in the field now, I thought I would take up the tourch and try to help others discover themselfs and the universe around them though this marvel of mankind, the simple machine capable of so much.

The Computer

\section*{About The Series}

The Fertile Framework series covers a wide range of topics related to technology and communication, from the basics of programming and hardware to advanced concepts in artificial intelligence and communication. Each book in the series provides a practical guide for readers who want to learn about modern computing and how it all works from the ground up. 
The series is meant to be a snapshot of documentation tools, exersises and commentary that is able to be a self-contained snapshot. Hosestly I have gotten a little annoyed with so much shift towards the cloud. That while very convisnent the default mode for software is to reach out to the internet as a general repository. And well this is really nice when it works but is by no means “self-reliant”. In the Linux and other open sources spaces its not like “they” prupusly hold back the ablity to host and distrubute your own COMPLETE set of software, there are tools and tuturials to become a mirror but this is not the DEFAULT. Instead you can find yourself trying to run something a little orlder for one reason or another and all the defaults assume you are using the most current OS and have a connection to the internet. 
What happens if time, censorship, politics … or something as simple as a bankrupsy or lack of intrest orphands a project and breaks the build process requiring the uninisated to embark on a dependancy hell that may have no real end. Instead some projects like “Linux From Scratch” or LSF. Address this head on, forcing you to go out and collect the software for your self. Well this sereies is ment to take this one step further in curiating all the collections of software, hardware specs, datasheets, howtos etc. and then providing paths for future learners. Regardless of a post-applolipic future or simply one where the self starter doesn’t have to worry about broken links or expired GPG keys, obsticals that can and should be avoided. 
I will generally be covering the last decade plus or minus for a few reasons. One is that the fundimentals and generally software has not radically changed about about the last 15 years. Thats about the time that affordable mulicore 64bit capablie machines hit the market and embeded really took off and some standards emerged. Piror to about 2005 the edge cases where abound and many manufacutures only released binary blobs as drivers, did’t document them and where not supported by the open source community. But as open source software (OSS) really took hold, and most of the manufacurers own engenners wanted to make their hardware work with OSS. We now have a wonderful ecosystem that feels like a human tresure, something that shouldbe free and accessable to all as a fundimental right. This project aims to encaspulate and contrubute to a historical legacy that begain in the 1970s and truly has matured during the post 2005 era.  

he Computer – This book introduces the core concepts of modern computing and provides a vertical slice approach to learning about the digital universe.
Networking – This book delves into the world of networking and teaches readers how to set up their own networks.  It covers topics such as protocols, routers, switches, firewalls, and security.
Databases – This book explores the world of databases and teaches readers how to design, implement their own database systems. It covers topics such as SQL, relational databases, NoSQL, and “big data”.
Artificial Intelligence – This book introduces readers to the exiting world of artificial intelligence and machine learning. It covers topics such as neural networks, deep learning, natural language processing, computer visions, and more.

The Fertile Framework series is perfect for anyone who wants to learn about modern computing and technology in a practical and fun way. Whether you're a beginner looking to get started with computers or an experienced professional looking to expand your knowledge, these books have something for everyone.

\section*{Intended Audience}

Beginners, who want to learn about modern computing and how it all works from the ground up. This book is suitable for those who have some experience with computers but are looking for a deep understanding of the underlying principles and concepts. The book is written in an approachable and fun style, making it accessible to readers who may not have a technical background or experience with computers or programming.

\section*{Overview of This Book}

\begin{flushleft}
\textbf{Chapter 1, Computers} Yep just what it sounds like we are going to introduce the core concepts of the computer. Its basic blueprints can be found in devices large and small. We will introduce the basics of the Von Newman architecture. The blue print for all modern computers from the last 70+ years. While specifics in speed, size, IO (Input/Output) have changed dramatically, an understanding of this basic architecture will give you the necessary grasp on everything from mainframes, gaming consoles, laptops, smart phones, smart TVs, and even industrial controllers used in manufacturing, robotics, and latest generation AI (Artificial Intelligence). It all stems from a common blueprint.\newline

\textbf{Chapter 2, Hardware} Will introduce you to this editions tested and recommended hardware to step though the examples in the book with. \newline

\textbf{Chapter 3, Virtualization} Whether its in the cloud, a web browser, or on your desktop you will eventually be running on physical hardware. There are good reasons that go all the way back as the 1960s as to why you might virtualize a computer. This chapter will introduce you to the concept of vitalization, some tools, options, and alternatives to physical bare metal hardware for running the examples in this book. \newline

\textbf{Chapter 4, Operating Systems} for many systems this is the next layer in the stack. The operating system manages the hardware and software resources, abstracting and delegating common services for programs to run, While not all computers require a distinct operating system most do. We will be focusing on Linux or more specifically GNU + Linux as it is not only the most democratized operating system, it is powerful and scales from the smallest IoT (Internet of things) sensors and switches, to large IBM Z Series mainframes. Unix the operating system that Linux models is the basis for Apple iOS and MacOS, Even the proprietary Microsoft Windows witch is POSIX compliant includes many approaches and even exact code that was created adapted from its original open source GNU roots. We will take a practical approach to installing a basic Linux system. \newline

\textbf{Chapter 5, Programming}, With working hardware and operating system we will jump into how you may write instructions for your computer to execute. We will brave though an introduction to the lowest level of human readable code called assembly language. Spend some time on high level programming using the god language C and its super set C++, Lastly we will snake though an interpreted language called Python. We will use some games and other fun examples along the way. Honorable mentions will be BASIC, Lisp, and Smalltalk … and ok Java...darn it, If we are going todo Java I will throw in JavaScript too.\newline

\textbf{Chapter 6, Synthesis}, the combination of ideas to form a theory or system. Bring together your learning in some musing cookbook examples to try. Such as a boot sector virus game, a pixel art program, and a program that simulates life itself.
\end{flushleft}

Note in the above list we will not be dedicating a specific chapter to Networking. That is the process of connecting two or more computers together to exchange information. While this is a very important function to learn, we will save diving into it for another day and instead focus on stand alone computing, while briefly menting networking at a high level throughout the book.

