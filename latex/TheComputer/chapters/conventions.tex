\section{Conventions}

This is the part of the book where we tell you about all the “conventions” like code listing, notes, etc. etc. But I will let you in on a little secret its as much for the writer as it is for you. See as a writer in a word processor, it helps us keep formatting consistent. Providing a single location to go back to and reference special formatting and hopefully just allow us to copy and paste the formatting. 
The following typographical conventions are used in the text of this book:
Italic
Indicates emphasis, new terms, URLs, email addresses, filenames, paths and Unit utilities.
Constant-width regular
Indicates commands, variables, attributes, functions, methods, modules, values, the contents of files, the output from commands, and code examples.
Constant-width bold
Indicates commands and other text that should be typed literally by the user, and code examples that demonstrate.
Constant-width Italic
Indicates commands and other text that should be typed literally by the user, and code or command examples.

\section{Code Examples}

The following is an example of a code listing. Also this section should talk about the code, scripts and commands and that they may be used or reproduced. They are syntax highlighted where possible and line numbered as appropriate. Just as a side note, if copying and pasting or otherwise reusing the contents of a listing, it should be understood that the line numbers should not be transferred and are for reference only.

 1  // Simple C program to display "Hello World"
 2  
 3  // Header file for input output functions
 4  #include <stdio.h>
 5  
 6  // main function -
 7  // where the execution of program begins
 8  int main()
 9  {
10  
11  	// prints hello world
12  	printf("Hello World");
13  
14  	return 0;
15  }


