\documentclass[12pt,A5]{book}
% letter (8.5x11) and A4 (8.4x 11.7) are close but not quite.
% While A5 seems to be a good journal size
% You fav moleskine is "large" 5x8.25
\usepackage[utf8]{inputenc}
\usepackage{amsmath}
\usepackage{amsfonts}
\usepackage{amssymb}

%% To change Sections?
\usepackage[T1]{fontenc}
\usepackage{titlesec}

%% Using Titlesec allows us to modify the chapters and sections to allow for stylazation without needing to modify the class
\titleformat
{\chapter} % command
[display] % shape
{\bfseries\Large\itshape} % format
{Chapter \ \thechapter} % label
{0.5ex} % sep
{
    \rule{\textwidth}{1pt}
    \vspace{1ex}
    \centering
} % before-code
[
\vspace{-0.5ex}%
\rule{\textwidth}{0.3pt}
] % after-code


\author{Andrew Brady}
\title{Hello Typeseting!}
\begin{document}
\maketitle
\tableofcontents

\chapter{Hello World!}

\section{Introduction}
Hello World!!! That is generally how you get started in programming. While understanding how to make the typeset documents of our dreams. We will need to tackel a few things past that in this first chapter. \newline But what about a code listing?

\begin{verbatim}
10 PRINT "HELLO WORLD!"
20 GOTO 10
\end{verbatim}

That was a simple \verb|BASIC| program, did you like it? With a simple code listing out of the way I do think we can get started. Keep in mind that there is no syntax highlighting there, but it will be covered in future sections.

\section{Specialization}
Moving allong with things, lets consider what we need to be able to cover. Prehaps with a list
\begin{quote}
"With Specialization, to a specifc enviroment a system will lose its ablity to adapt"
\end{quote}
That is one of my favorate quotes from evloutionary biology. But I first saw it in \textit{The Phcology of a Programmer}

\chapter{Moving Along...}

What happens if we put a little something here kinda like a summary at the beginning of a chapter

\section{Frist Princaples}

Let's discuss what "Frist Princaples" really means.

\end{document}